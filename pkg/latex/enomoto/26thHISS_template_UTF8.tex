%%%%%%%%%%%%%%%%%%%%%%%%%%%%%%↓設定↓%%%%%%%%%%%%%%%%%%%%%%%%%%%%%
%このフォーマットは第26回HISS査読用論文フォーマットです.
%同封の26thHISS.styも作業フォルダにコピーしてお使いください.
\documentclass[a4j,twocolumn,10pt]{jarticle}
\usepackage{26thHISS}
\headsep=0mm
\textheight=24cm
%%%%%%%%%%%%%%%%%%%%%%%%%%%%%%↑設定↑%%%%%%%%%%%%%%%%%%%%%%%%%%%%%

\title{ 第26回IEEE広島支部学生シンポジウム論文フォーマット\\
The 26th IEEE Hiroshima Section Student Symposium \\
Papers Format
}

\author{%
電気 太郎 $^{\dagger}$ 情報 花子 $^{\dagger\dagger}$ \\
Taro Denki $^{\dagger}$ Hanako Joho $^{\dagger\dagger}$\\
$^{\dagger}$HISS大学 工学部 \hspace*{1em}
 $^{\dagger\dagger}$HISS大学 情報工学部 %\\
}

\begin{document}
\maketitle
\thispagestyle{empty} %%←消さない%%
\pagestyle{empty} %%←消さない%% 追加 2006.10.13
\baselineskip=4.5mm %%←消さない%%

\section{概要}
これは第26回IEEE広島支部学生シンポジウムの論文フォーマットである.
論文の原稿を執筆する際は,このフォーマットに従うこととする.
なお,英語論文は同様の形式だが英語表記のみとする.

\section{スタイル}
原稿はA4判を用いることとする.
余白は上部20mm,下部20mm,左右20mmにし,内部領域に原稿が収まるようにする.
テキストコードはUTF--8を使用する.

\subsection{表題部}
表題は見やすくするために大きなフォントを使用することとする.
表題の文字の大きさは14--16ptを使用することとする.
著者名等は10ptを用いる.大きいフォントがないときは太字のフォントを使用する.
表題および著者名は和文・英文両方を記す.
所属機関名は和名のみを記す.
著者が複数で所属がそれぞれ異なる場合もそれぞれ記す.

\subsection{本文}
\begin{enumerate}
\item 本文のフォントの大きさは9--10ptとする.
\item 本文は2段組とし,ページ数は原則2--8ページ(両面刷り1--4枚)とする(ページ番号は入れないこと).
\item 本文は「である」調とし,できるだけ平易に表現する.
専門用語以外は常用漢字を使用する.句読点は「,」「.」に統一する.
\item 全体を通して用語を統一する.
\item 略語に関しては,( )内に名称を記載する.また,脚注を使用してもよい.
\item 数字は原則としてアラビア数字を用いるが,文章になっている場合は漢数字を用いてもよい.(数100m→数百m)
\end{enumerate}
\vspace{-3mm}
\subsection{図,表および写真}
\begin{enumerate}
\item 図表はA4判の紙1枚当たり1, 2点を目安とし,鮮明に描く.また,図および表には表題,通し番号をつける.
\vspace{-2mm}
\item 図・写真は原稿をそのまま使用するので,大きめに描き,線の太さ,濃淡,文字の大きさに注意して作成する.
\vspace{-2mm}
\item グラフの縦軸,横軸には必ず軸の名称と単位を記入する.
\end{enumerate}
\vspace{-3mm}
\subsection{文献}
\begin{enumerate}
\item 引用文献は本文の登場順に通し番号を付け,本文中該当部分に~\cite{b1}等のように印をつける.
\vspace{-2mm}
\item 1文献につき1番号を対応させる.同一著者の別の文献は別番号とする.
\vspace{-2mm}
\item 文献の引用にあたっては必ず出典を明記し,必要に応じて原著者の了承を得る.
\vspace{-2mm}
\item 文献の著者名は日本語ならばフルネームで示し,英語ならば名前はイニシャルで書く.
\end{enumerate}
\vspace{-3mm}
\section{特許に関する注意}
本シンポジウムは,特許法第30条第1項(発明の新規性の喪失の例外)の対象とならない.
このため,本シンポジウムで発表する内容を特許申請する予定のある方は,
論文集の発行日(2024年11月25日)より前に特許申請を行っていただきたい.

\section{査読用論文の提出先}
第26回HISS ホームページの査読用論文提出フォームの投稿システムにアクセスして,連絡先の住所,所属,名前,投稿論文の分野,等必要事項を記入し,原稿PDF ファイルを2024年08月19日(月) 17:00までにアップロードする.
投稿システムの利用が難しい場合は,下記E-mailへ.\\
\section{問い合わせ先}
質問等は下記までお問い合わせ願います.
\\
――――――――――――――――――――――――
\\\small{〒731-3194\\}
\small{広島県広島市安佐南区大塚東3丁目4番10号\\}
\small{広島市立大学 情報科学部 情報工学科 情報ネットワーク研究グループ 内\\} 
\small{(第26回HISS論文TP委員長 原 惇樹)}
\\
\\\small{E-mail:hiss\_tp@hiss26th.sakura.ne.jp}
\\\small{第26回HISS HP: https://hiss26th.sakura.ne.jp/}
\small{}
\\ 
%参考文献
\begin{thebibliography}{9}
\bibitem{b1}
T.Denki,H.Joho,"26thHISS,"\textit{IEEE},Vol.1,No.1,pp.1-10,2000.
\end{thebibliography}
\end{document}
