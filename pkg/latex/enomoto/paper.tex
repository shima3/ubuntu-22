%%%%%%%%%%%%%%%%%%%%%%%%%%%%%%↓設定↓%%%%%%%%%%%%%%%%%%%%%%%%%%%%%
%このフォーマットは第26回HISS査読用論文フォーマットです.
%同封の26thHISS.styも作業フォルダにコピーしてお使いください.
\documentclass[a4j,twocolumn,9pt]{jarticle}
\usepackage{26thHISS}
\usepackage{ascmac}
\usepackage {diagbox}
\headsep=0mm
\textheight=24cm
%%%%%%%%%%%%%%%%%%%%%%%%%%%%%%↑設定↑%%%%%%%%%%%%%%%%%%%%%%%%%%%%%

\title{ ChatGPTを用いたプログラム生成におけるプログラミング言語間の精度比較\\
Comparison of accuracy between programming languages in program generation using ChatGPT
}

\author{%
榎本 創太 $^{\dagger}$ 島 和之 $^{\dagger}$ \\
Sohta Enomoto $^{\dagger}$ Kazuyuki Shima $^{\dagger}$\\
$^{\dagger}$ 広島市立大学 情報科学研究科 \hspace*{1em}
%\\
}

\begin{document}
\maketitle
\thispagestyle{empty} %%←消さない%%
\pagestyle{empty} %%←消さない%% 追加 2006.10.13
\baselineskip=4.5mm %%←消さない%%

\section{まえがき}
% 生成AIの発展
ChatGPTをはじめとした生成系AIサービスが登場し, 幅広い分野で利用され始めている.
これらのサービスは個人のPCやスマートフォンで手軽に利用することができるため, 創作に必要な知識や技術を持たないユーザであっても, 文章や画像を生成することが可能となった.
しかし, ChatGPTの回答は大量の学習データから深層学習によって構築した言語モデルから統計的にそれらしい応答が出力されるため, 出力される回答には誤りが含まれている可能性が常に存在する. 
% 文部科学省の方針
参考文献~\cite{Monkasyo2023}は文部科学省が発行した初等中等教育における生成AI利用のガイドラインであり, 生成AIの教育利用について述べられている. 
文書中では, 生成AIは新たな情報技術であり, 生産性の向上に活躍している生成AIの仕組みを理解し, 将来的に使いこなすための力を育てる必要があるとする一方, 利用時の危険性, 創造性や学習意欲への影響を考慮し, 限定的な利用から始めることが適切であるとしている. 
% 生成AIを利用したプログラミング学習
また, ChatGPTは文章だけでなく, プログラムを生成することも可能なため, プログラミング学習にも活用される.
ChatGPTは, 学習者にサンプルプログラムの生成, プログラム解説, プログラム修正等を提供する. 
しかし, ここで出力される回答も全て正しい情報であるとは限らない. 
学習段階のユーザにとって, 回答に含まれる誤りを見つけることは非常に困難である. 

% 目的 要検討
ChatGPTによるプログラム生成の不得意分野を把握することで,学習者にとって出力されたプログラムの信用性を判断する有益な材料となることが期待できる.
そこで本研究では, ChatGPTによるコード生成性能向上のための初期調査として,ChatGPTによるプログラミング言語間のプログラム生成精度について調査する. 

\section{関連研究}
% ChatGPTによるプログラミング授業の課題の解答生成の評価
参考文献~\cite{Suzuki2023}では, ChatGPTが大学のプログラミング授業(Python言語)に及ぼす影響を評価している. 
実験では, 大学1年生を対象とした2つのプログラミング授業(基礎,応用)の授業課題をChatGPTにプロンプトとして与え, 出力されたプログラムを編集することなく受講者が提出すると想定し, どの程度正解できるか調査している. 
また, ~\cite{Suzuki2023}ではChatGPTを用いて解くのが難しい課題の傾向として, 入出力トークン数に着目し考察されている. 

実験の結果, 基礎授業課題ではChatGPT-3.5を用いて91\%(=32/35), 応用授業課題では35\%(6/17)の課題で正解と判定された. 
応用問題についてChatGPT-4を用いて追加実験を行った結果,76\%(13/17)の課題で正解と判定された. 

実験の結果から~\cite{Suzuki2023}では, ChatGPT-4を用いれば, どちらの授業であっても十分に合格することができると結論づけられている. 
また,ChatGPT-4でのみ正解と判定された課題と両モデルで正解と判定された課題の累計出題長や解答長に明確な差はないとされ, 両モデルで非正解と判定された課題はChatGPT-4のトークン制限が影響したのではないかと述べられている.

参考文献~\cite{Koyanagi2024}では,プロンプトとして日本語,英語,中国語のデータセットを使用して,GitHub Copilotによるコード生成を行っている.
実験の結果,日本語,英語,中国語の順に正答率が高く,英語と中国語には基礎問題において約 11.5\%の差が生じたことが確認されている.

\section{実験}
\subsection{実験方法}
本研究では, 関連研究の結果を踏まえ, Python以外のプログラミング言語でプログラム生成を行い, 正答率にどのような違いがあるのかを調査する.

実験を行うにあたって,次のような条件を設定し実験を行った.
\begin{itemize}
\item 利用したChatGPTのモデル : ChatGPT-3.5(2023/9/28時点のもの)
\item 使用問題 : 競技プログラミングの鉄則 演習問題集 A問題 66問
\item 使用言語(入力) : 日本語
\item 使用言語(出力) : C++, Java
\item 採点に利用したサイト : AtCoder
\end{itemize}

AtCoderは競技プログラミングコンテストを開催しているWebサイトであり, 演習問題が公開されている. 
本実験ではAtCoder内で提供されている競技プログラミングの鉄則演習問題と対応する自動採点システムを利用した.
演習問題には問題文, 制約条件, 入出力形式, 入出力例が与えられる.
これをテキスト形式で記述しChatGPTにプロンプトとして与え, 出力したプログラムを編集することなくAtCoderの自動採点システムに提出し採点を行った.

また, 使用した問題集の問題は大きく9つのテーマに問題が分類されており,ChatGPTが苦手とする分野の調査も重ねて行った. 
\subsection{前提条件}
% 実験前提条件
実験を行うにあたって次のような前提条件を設定を行った. 
\begin{itemize}
  \item ChatGPTは同じ入力であっても, 出力が異なるため一つの問題につき5回程度プログラム生成を行い, 一度でも正解と判定された問題を正解として扱う. 
  \item ChatGPTは前の入力を引き継いで出力文を生成するため, 問題ごとに新規チャットを立ち上げ実行する. 
  \item ChatGPTがテキスト入力しか受け付けないため, 問題に図や表が含まれる場合, テキストで表現できるものはテキストに変換し, そうでないものは省略し実行した. 
  \item AtCoderの採点結果は正解, 不正解, コンパイルエラー, 実行時エラー, 実行時間超過の五種類に判別される. 実行時間超過と判定された問題について, テストケースのサイズによって判定が変化したため, 本実験では, 正解と判定された問題のみを正解とし, 他の4つを非正解として扱う. 
\end{itemize}
\subsection{実験結果}
実験の結果は次のようになった. 
C++では45.5\%(30/66),  Javaでは48.5\%(32/66)の問題で正解と判定された. 
不正解と判定された問題には, アルゴリズムや計算式が適切でない, 出力形式が条件を満たしていない, 例外処理がされていないなど様々だった. 
両言語とも正解と判定された問題は, 34.8\%(23/66)であった. 
\subsection{考察}
関連研究の結果とは異なり両言語とも正答率50\%を切る結果となった. 
このような結果となった理由の一つとして, 使用問題の性質が影響したのではないかと考えられる.
ChatGPTはパターンマッチで出力する内容を決定しているため, 資料の少ないものの回答には不正確なものが多く含まれる傾向にある.  
競技プログラミングの問題は性質上, 競技者自身でアルゴリズムを考え解くよう工夫されているため, ChatGPTがプログラム生成する際, パターンマッチで適合したものが少なかったと考えられる. 
実際, 上記の実験で不正解と判定された問題の多くは, 考察テクニック, グラフアルゴリズムに分類されるオリジナル性の高い問題だった.

本研究は, ChatGPTを用いたプログラム生成の言語間性能調査を目的とするため,C++とJAVAの両言語で正解と判定された問題に着目し,さらに別の言語で追加の実験を行った.

\subsection{追加実験}
% 別言語での調査
自動採点システムに対応した言語のうち, 検索エンジンのデータを基にしたプログラミング言語ランキング「TIOBE」のランキング30位以下(2023年11月時点)の6言語で同様の実験を行った. 
\begin{itemize}
    \item 利用したChatGPTのモデル : ChatGPT-3.5(2023/10/16時点のもの)
    \item 使用問題 : 両言語とも正解と判定された問題5問[表\ref{mondai}]
    \item 使用言語 : Lua, Dart, Julia, Haskell, Crystal, なでしこ

\end{itemize}

\begin{table}[h]
    \caption{使用問題内容}
    \label{mondai}
    \centering
    \begin{tabular}{l|l}
    \hline
    言語  & 問題内容\\ 
    \hline
    A01 & 入力された自然数の平方数を出力\\
    A04 & 入力された自然数を2進数へ変換し出力\\
    A11 & 配列の要素を二分探索\\
    A19 & ナップサック問題\\
    A27 & 入力された2つの自然数の最大公約数を出力\\
  \hline
  \end{tabular}
\end{table}

\subsection{追加実験結果}
実験の結果は[表\ref{kekka}]のようになった. 
Dartのみ全問正解と判定され,  Lua, Julia, Haskellはそれぞれ4問ずつ正解と判定された. 
Crystal, なでしこは一問も課題を満たすプログラムを生成しなかった. 
Crystalについてプログラムのエラー文をChatGPTに引き続き与えると正解と判定されるプログラムを出力したが, なでしこは正解と判定されるプログラムを生成しなかった.

\begin{table}[h]
  \caption{実験結果}
  \label{kekka}
  \centering
\begin{tabular}{c|ccccc}
  \hline
  \diagbox{言語}{問題} & A01 & A04 & A11 & A19 & A27\\
  \hline
  Lua & $\bigcirc$ & $\times$ & $\bigcirc$ & $\bigcirc$ & $\bigcirc$\\
  Dart & $\bigcirc$ & $\bigcirc$ & $\bigcirc$ & $\bigcirc$ & $\bigcirc$\\
  Julia & $\bigcirc$ & $\bigcirc$ & $\bigcirc$ & $\times$ & $\bigcirc$\\
  Haskell & $\bigcirc$ & $\bigcirc$ & $\bigcirc$ & $\times$ & $\bigcirc$\\
  Crystal & $\times$ & $\times$ & $\times$ & $\times$ & $\times$\\
  なでしこ & $\times$ & $\times$ & $\times$ & $\times$ & $\times$\\
  \hline
\end{tabular}
\end{table}

\section{ChatGPTが生成したなでしこプログラム}
プログラミング言語「なでしこ」は日本語で記述する日本語プログラミング言語である. 
以下はChatGPTに最大公約数を求める問題を与えた際に出力したプログラムの関数部分を抜き出したものとなでしこで記述したプログラムである. 
\begin{screen}
  FUNC 最大公約数(整数A, 整数B)\\
  \qquad IF B == 0\\
  \qquad \qquad RETURN A\\
  \qquad ELSE\\
  \qquad \qquad RETURN 最大公約数(B, A \% B)\\
  \qquad ENDIF\\
END FUNC
\end{screen}
\begin{screen}
●最大公約数(Aの,Bの)\\
\qquad もし,Bが0ならば\\
\qquad \qquad Aで戻る\\
\qquad 違えば,\\
\qquad \qquad 最大公約数(B,A\%B)で戻る\\
\qquad ここまで\\
ここまで
\end{screen}

アルゴリズム自体は最大公約数を求めるものであるが, なでしこでの記述方式がChatGPTの学習内容に含まれていないためか, ChatGPTはなでしこの文法ではなく, 最大公約数を求める疑似コードを出力した. 
% この結果より, 出力するプログラムの文法を変更することは, ChatGPTのプログラム生成の阻止に一定の効果があると言える. 
出力されたコードとなでしこで記述されたプログラムに明確な差があるため, プログラム初学者であっても間違った出力がされているとわかりやすい.
しかしこれは,珍しいケースであり非正解と判定されたプログラムの多くは使用言語形式に沿っているように見えるコードであり, どこにエラーが含まれているか判別するのは困難であった.

\section{まとめと今後の予定}
本研究では, ChatGPTによるコード生成性能向上のための初期調査として, ChatGPTによるプログラミング言語間のプログラム生成精度比較についての調査を行った.
実験の結果, ChatGPTのプログラム生成には生成する言語によって生成能力に差があり, 不正解と判定された問題の多くは, 考察テクニック, グラフアルゴリズムに分類されるオリジナル性の高い問題だった.
有名でない言語を用いてプログラム生成をした際には, 指定言語の代わりに擬似コードが出力された. 
今後の予定として, ChatGPTに与えるプロンプト形式の変更によるプログラム性能調査が挙げられる. 

%%%%%%%%%%%%%%%%%%%%%%%%%%%%%%%%%%%%%%
% \section{概要}
% これは第26回IEEE広島支部学生シンポジウムの論文フォーマットである.
% 論文の原稿を執筆する際は,このフォーマットに従うこととする.
% なお,英語論文は同様の形式だが英語表記のみとする.

% \section{スタイル}
% 原稿はA4判を用いることとする.
% 余白は上部20mm,下部20mm,左右20mmにし,内部領域に原稿が収まるようにする.
% テキストコードはUTF--8を使用する.

% \subsection{表題部}
% 表題は見やすくするために大きなフォントを使用することとする.
% 表題の文字の大きさは14--16ptを使用することとする.
% 著者名等は10ptを用いる.大きいフォントがないときは太字のフォントを使用する.
% 表題および著者名は和文・英文両方を記す.
% 所属機関名は和名のみを記す.
% 著者が複数で所属がそれぞれ異なる場合もそれぞれ記す.

% \subsection{本文}
% \begin{enumerate}
% \item 本文のフォントの大きさは9--10ptとする.
% \item 本文は2段組とし,ページ数は原則2--8ページ(両面刷り1--4枚)とする(ページ番号は入れないこと).
% \item 本文は「である」調とし,できるだけ平易に表現する.
% 専門用語以外は常用漢字を使用する.句読点は「,」「.」に統一する.
% \item 全体を通して用語を統一する.
% \item 略語に関しては,( )内に名称を記載する.また,脚注を使用してもよい.
% \item 数字は原則としてアラビア数字を用いるが,文章になっている場合は漢数字を用いてもよい.(数100m→数百m)
% \end{enumerate}
% \vspace{-3mm}
% \subsection{図,表および写真}
% \begin{enumerate}
% \item 図表はA4判の紙1枚当たり1, 2点を目安とし,鮮明に描く.また,図および表には表題,通し番号をつける
% \vspace{-2mm}
% \item 図・写真は原稿をそのまま使用するので,大きめに描き,線の太さ,濃淡,文字の大きさに注意して作成する.
% \vspace{-2mm}
% \item グラフの縦軸,横軸には必ず軸の名称と単位を記入する.
% \end{enumerate}
% \vspace{-3mm}
% \subsection{文献}
% \begin{enumerate}
% \item 引用文献は本文の登場順に通し番号を付け,本文中該当部分に~\cite{b1}等のように印をつける.
% \vspace{-2mm}
% \item 1文献につき1番号を対応させる.同一著者の別の文献は別番号とする.
% \vspace{-2mm}
% \item 文献の引用にあたっては必ず出典を明記し,必要に応じて原著者の了承を得る.
% \vspace{-2mm}
% \item 文献の著者名は日本語ならばフルネームで示し,英語ならば名前はイニシャルで書く.
% \end{enumerate}
% \vspace{-3mm}
% \section{特許に関する注意}
% 本シンポジウムは,特許法第30条第1項(発明の新規性の喪失の例外)の対象とならない.
% このため,本シンポジウムで発表する内容を特許申請する予定のある方は,
% 論文集の発行日(2024年11月25日)より前に特許申請を行っていただきたい.

% \section{査読用論文の提出先}
% 第26回HISS ホームページの査読用論文提出フォームの投稿システムにアクセスして,連絡先の住所,所属,名前,投稿論文の分野,等必要事項を記入し,原稿PDF ファイルを2024年08月19日(月) 17:00までにアップロードする.
% 投稿システムの利用が難しい場合は,下記E-mailへ.\\
% \section{問い合わせ先}
% 質問等は下記までお問い合わせ願います.
% \\
% ――――――――――――――――――――――――
% \\\small{〒731-3194\\}
% \small{広島県広島市安佐南区大塚東3丁目4番10号\\}
% \small{広島市立大学 情報科学部 情報工学科 情報ネットワーク研究グループ 内\\} 
% \small{(第26回HISS論文TP委員長 原 惇樹)}
% \\
% \\\small{E-mail:hiss\_tp@hiss26th.sakura.ne.jp}
% \\\small{第26回HISS HP: https://hiss26th.sakura.ne.jp/}
% \small{}
% \\ 
%参考文献

%%%%%%%%%%%%%%%%%%%%%%%%%%%%%%%%%%%%%%

\begin{thebibliography}{9}
\bibitem{Monkasyo2023}
文部科学省初等中等教育局,"初等中等教育段階における生成AIの利用に関する暫定的なガイドライン"\textit{文部科学省初等中等教育局},2023.
\bibitem{Suzuki2023}
鈴木智也,神谷年洋,"ChatGPTによるプログラミング授業の課題の解答生成の評価"\textit{信学技報},Vol.123,No.123,pp.55-60,2023.
\bibitem{Koyanagi2024}
小柳慶,野口広太郎,王棟,近藤将成,亀井靖高,鵜林 尚靖,"GitHub Copilotを用いたコード推薦における入力言語の影響調査"\textit{FOSE2023},2023.
\end{thebibliography}

\end{document}
