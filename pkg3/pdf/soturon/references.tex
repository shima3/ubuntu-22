  \bibitem{mlit}
  国土交通省:
  令和4年の土砂災害発生件数は795件,
  砂防NEWS Press Release,(2023年3月).

  \bibitem{zentyou}
  国土交通省水管理・国土保全局砂防部 : 
  「土砂災害警戒避難に関わる前兆現象情報の活用のあり方について」, 
  \url{http://www.mlit.go.jp/common/001021004.pdf}, (2006年3月). 



  \bibitem{watanabe}
  渡邊 康平:
  土砂災害の前兆現象検知を目的とした画面分割と深層学習を用いた水位変動の推定,
  広島市立大学大学院情報科学研究科システム工学専攻修士論文,
  (2023年1月). 

  \bibitem{seman}
  Nur Atirah Muhadi ,Ana Mijic , Ahmad Fikri Abdullah ,
  Siti Khairunniza Bejo , Muhammad Razif Mahadi: 
  Deep Learning Semantic Segmentation for Water Level Estimation Using Surveillance Camera,
  Applied Sciences,vol.11,issue.20,(2021).

  \bibitem{deeplabv3+}
  Chen, L. C., Zhu, Y., Papandreou, G., Schroff, F.,Adam, H. :
  Encoder-decoder with atrous separable convolution for semantic image segmentation. 
  In Proceedings of the European conference on computer vision (ECCV) (pp. 801-818).(2018).

  \bibitem{segnet}
  Badrinarayanan, V., Kendall, A.,Cipolla, R.:
  Segnet: A deep convolutional encoder-decoder architecture for image segmentation. 
  IEEE transactions on pattern analysis and machine intelligence, 39(12), 2481-2495.(2017).

  \bibitem{IoU}
  青島亘佐, 山本拓海, 中野聡,中村秀明:
  深層学習によるセグメンテーション手法を用いたコンクリート表面の変状領域の検出.
  AI・データサイエンス論文集, 1(J1), 481-490.(2020年).
  
  \bibitem{bf}
  山根達郎,全邦釘: 
  Deep learning による Semantic Segmentation を用いたコンクリート表面ひび割れの検出. 
  構造工学論文集 A, 65, 130-138.(2019年).
