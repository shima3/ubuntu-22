%ここから
% \documentclass[12pt,a4j]{jarticle}
\documentclass[12pt,a4j,uplatex]{jarticle} % 縦書き用
\usepackage[dvipdfmx]{graphicx}
\usepackage{float}
\usepackage{times}
\usepackage{amsmath}
\usepackage{url}
\usepackage[titletoc,title]{appendix}
\usepackage{plext} % 縦書き用
\usepackage[a4paper]{geometry} % 縦書き用

\pagestyle{plain}
\setlength{\textwidth}{150mm}
\setlength{\textheight}{230mm}
\setlength{\oddsidemargin}{+4.6mm}
\setlength{\topmargin}{-10mm}
\setlength{\footskip}{20mm}
\newfont{\ssf}{cmssbx10 scaled\magstep3}

\nonstopmode
%日本語用に jplain 英語の場合は plain にすること
% \bibliographystyle{jplain}
\bibliographystyle{esdhcu}

%何か追加したい場合,ここから追加すること
\usepackage{newenum}
\usepackage{listings}
%\usepackage[dvips]{graphicx}
\usepackage[dvipdfmx]{graphicx,xcolor}
\usepackage[T1]{fontenc}
\usepackage{lmodern}
%\usepackage{textcomp}
\usepackage{latexsym}
%\usepackage[fleqn]{amsmath}
%\usepackage{amssymb}
\usepackage{comment}
\usepackage{fancyvrb}
\usepackage{multicol}
\usepackage{newtxtext}
\usepackage{enumerate}
\usepackage{wrapfig}

%\setlength{\topskip}{0cm} % 段落間
%\setlength{\partopsep}{0cm} % 項目間

\makeatletter
\renewenvironment{itemize}
  {\ifnum \@itemdepth >\thr@@\@toodeep\else
   \advance\@itemdepth\@ne
   \edef\@itemitem{labelitem\romannumeral\the\@itemdepth}%
   \expandafter \list \csname \@itemitem\endcsname{%
      \iftdir
         \ifnum \@listdepth=\@ne \topsep.5\normalbaselineskip
           \else\topsep\z@\fi
         \parskip\z@ \itemsep\z@ \parsep\z@
         \labelwidth1zw \labelsep.3zw
         \ifnum \@itemdepth =\@ne \leftmargin1zw\relax
           \else\leftmargin\leftskip\fi
         \advance\leftmargin 1zw
      \fi
         \def\makelabel##1{\hss\llap{##1}}}%
   \fi}{\endlist}

\renewenvironment{description}
  {\list{}{\labelwidth\z@ \itemindent-\leftmargin
    \iftdir
    \leftmargin\leftskip \advance\leftmargin3\Cwd
    \rightmargin\rightskip
    \labelsep=0zw \itemsep\z@
    \listparindent\z@ \topskip\z@ \parskip\z@ \partopsep\z@
    \fi
    \let\makelabel\descriptionlabel}
    % \setlength{\parsep}{0mm} % 各段落間のスペース
    % \setlength{\parskip}{0mm} % 段落の前に入る縦方向スペース
    % \setlength{\itemsep}{0mm} % 項目間に入る縦方向スペース
  }{\endlist}
\renewcommand{\descriptionlabel}[1]{%
  \hspace\labelsep\normalfont\bfseries #1}
\makeatother

% \setlength{\itemindent}{0em}
% \setlength{\leftskip}{0em}
% \setlength{\listparindent}{-1em}

\newenvironment{dl}{
  % \setlength{\leftmargin}{0em}
  % \setlength{\leftskip}{0em}
  \begin{description}
    \setlength{\itemindent}{-1em}
    \setlength{\leftskip}{0em}
    % \addtolength{\itemindent}{1em}
    % \addtolength{\leftskip}{1em}
    \setlength{\parskip}{0mm} % 段落の前に入る縦方向スペース
    \setlength{\itemsep}{1mm} % 項目間に入る縦方向スペース
}{\end{description}}

\newenvironment{ol}{
  \begin{enumerate}
    \setlength{\itemindent}{0em}
    \setlength{\leftskip}{1em}
    % \addtolength{\itemindent}{-1em}
    % \addtolength{\leftmargin}{1em}
    % \setlength{\parsep}{0mm} % 各段落間のスペース
    \setlength{\parskip}{0mm} % 段落の前に入る縦方向スペース
    \setlength{\itemsep}{0mm} % 項目間に入る縦方向スペース
}{\end{enumerate}}

\newenvironment{ul}{
  \begin{itemize}
    \setlength{\itemindent}{0em}
    \setlength{\leftskip}{0em}
    % \addtolength{\leftmargin}{1em}
}{\end{itemize}}

\emergencystretch=1em

\renewcommand{\lstlistingname}{リスト}
\lstset{language=,%
        basicstyle=\ttfamily\footnotesize,%
        commentstyle=\textit,%
        classoffset=1,%
        keywordstyle=\bfseries,%
	% frame=single,
	frame=tRBl,
        framesep=5pt,%
	showstringspaces=false,%
        numbers=left,
        stepnumber=1,
        numberstyle=\footnotesize%
	}%

% 図と図の間のスペース
\setlength\floatsep{2truemm}
% 本文と図の間のスペース
\setlength\textfloatsep{2truemm}
% 本文中の図のスペース
\setlength\intextsep{0pt}
% 図とキャプションの間のスペース
\setlength\abovecaptionskip{1truemm}

% ぶら下げ
\newenvironment{hang}[1][\parindent]
  {\def\item{\par\hangindent=#1\noindent}}
  {\par}

% カンマの後,改行を許す
% \def\,{,\allowbreak}
% これを定義すると \cite に余計なカンマが入る.

% 縦棒の後,改行を許す
\def\|{\tt |\allowbreak}
% circumflex accent, caret, up arrow, hat, chevron
\def\hat{\mbox{\^{ }}}
% 鍵括弧
\def\sb#1{[#1]}
% 二重鍵括弧
\def\db#1{[\![#1]\!]}
% 三重鍵括弧
\def\tb#1{[\![\![#1]\!]\!]}
% 角括弧
\def\sa#1{\langle#1\rangle}
% 二重角括弧
\def\da#1{\langle\!\langle#1\rangle\!\rangle}
% 矢印
\def\->{\allowbreak\mbox{$\rightarrow$}\allowbreak}
% ・・・
\def\c...{\mbox{$\cdots\mbox{}$}}
% :=
\newcommand{\defeq}{\mathrel{\mathop:}=}

\newcommand{\NT}[1]{\ensuremath{\langle\mbox{#1}\rangle}\allowbreak}
\newcommand{\C}[1]{\mbox{\tt#1}}

% Hatの意味関数
\newcommand{\sH}[1]{\ensuremath{\mathcal{H}}\dlrBrack{\mbox{\tt#1}}}
\newcommand{\dlrBrack}[1]{\ensuremath{[\![#1]\allowbreak\!]}}

% Schemeの意味関数
\newcommand{\sS}[1]{\ensuremath{\mathcal{E}}\dlrBrack{\mbox{\tt#1}}}

% 式に含まれる自由変数の集合
\newcommand{\FV}[1]{\ensuremath{\mathcal{F}}\dlrBrack{\C{#1}}}

% ドットの前後に空白
\newcommand{\D}{\mbox{\tt\ .\ }}

% 継続付き関数
\newcommand{\FC}[2]{\mbox{\tt F.C #1\D#2}}

%追加ここまで

%ここまで
